%!TEX root=main.tex

\section{Introduction} \label{intro}

The number and severity of cyber attacks has grown significantly in recent years~\cite{Symantec-Threat-Report,IBM-XForce-Report}. 
Cyber attackers have managed to pull off virtual bank heists, distributed denial of service (DDoS) attacks powered by botnets and Internet of Things (IoT) devices, and power outages caused by malware~\cite{IBM-XForce-Report}. 
Our National Science Foundation (NSF)--sponsored {\em SecureCloud} test environment aims to combat the growing number of cyber attacks against cloud networks using an autonomic, zero trust environment~\cite{7796146}.  

Our recent {\em SecureCloud} test environment implementation utilizes new software developed for use in an {\textbf O}bserve {\textbf O}rient {\textbf D}ecide {\textbf A}ct (OODA) control plane~\cite{something for OODA}.
Part of this system uses G-star, the Dynamic Graph Database~\cite{Labouseur-DAPD-2015} to organize, visualize, and analyze cyber attack data. 
G-star, interoperating with the PostgreSQL object relational database through G-star Studio~\cite{inroads-Labouseur16}, allows us to perform graph theoretical analysis and relational queries on cyber-attack data.  

Key contributions of this work include the following:
\begin{itemize}
   \item introduces the idea of an API honeypot
   \item describes a reference implementation of an API honeypot
   \item demonstrates DDoS and malware API attack analysis using graph and relational tools
   \item discusses performance characteristics of our API honeypot
   \item shows how we enable security experts to analyze data and develop remedies for emerging API attacks  
\end{itemize}

The remainder of this paper is organized as follows: 
Section~\ref{background} introduces the idea of an API honeypot. 
Section~\ref{construction} describes the software design and features of our API honeypot, Pasithea. 
Section~\ref{analysis} presents an analysis of data received from both the initial G-star REST API logs and current data collected by Pasithea. 
Section~\ref{performance} explains results from Pasithea performance testing.
Finally, Section~\ref{conclusions} ends this paper with a discussion of our initial conclusions and plans for future work.
