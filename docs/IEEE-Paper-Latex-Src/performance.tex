%!TEX root=main.tex

\section{Performance Tests} \label{performance}

Performance testing is a critical part of development, so it is important to demonstrate Pasitheas performance under different loads and its ability to log many, potentially thousands, of incoming requests in a short amount of time. In other words, it must respond fast enough to keep malicious users interested while also being stable enough to receive high volumes of incoming requests. To do this, we ran a series of benchmarks using the Apache Bench (ab) tool \cite{ab}. This tool allows us to designate a number of completed requests to be sent to our API honeypot while varying the number of simulated concurrent users. The results from these tests are displayed in Fig. \ref{fig:R/s} and Fig. \ref{fig:T/R}. We researched a baseline response time for a RESTful API to give this data appropriate context. In doing so, we discovered two separate internal tests from software development and web monitoring companies, 3PillarGlobal \cite{3Pillar} and Site24x7 \cite{site24x7}. Paired with some research on the human perception of performance \cite{performance}, we concluded that a 300-ms response time is expected under normal traffic conditions in order for the API honeypot to appear realistic. The data Pasithea collected indicates that we fall well within this range given a concurrency level of 500. In addition, we continued tests at much higher concurrency levels to assess how well Pasithea would perform under extreme stress, like the attempts we saw on the GStar API. Pasithea can, with time, handle a concurrency level over 9000 while still logging more than 90\% of the requests received.

\begin{figure}[ht]
\centering
\includegraphics[width=2.5in]{images/RequestsperSecond.png} 
\caption{-- Requests processed per second by Pasithea at varying concurrency levels.}
\label{fig:R/s}
\end{figure}

\begin{figure}[ht]
\centering
\includegraphics[width=2.5in]{images/TimeperRequest.png} 
\caption{-- Mean time taken to complete a single request on Pasithea at varying concurrency levels.}
\label{fig:T/R}
\end{figure}

Since our implementation of request/response for Pasithea was deliberately kept very simple (only responding with 404 errors), we have thus far been unsuccessful in driving Pasithea hard enough during performance testing to reach a point where it is unable to handle a significant amount of requests.  Pasithea hits a R/s plateau at a concurrency level of 1000, but continues to perform well at 9000. With enough storage space, we believe that Pasithea could withstand a substantial attack, such as the one seen on the GStar Graph Database, and be able to log information about the attack for further analysis.
